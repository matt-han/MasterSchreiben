\chapter{Systementwurf}\label{chp:systementwurf}


%#######################################################################################
%#######################################################################################
\section{Variantenmanagement und Parameter}
\paragraph{}

Die Lösung zum Speichern und Darstellen der Parameterwerte abhängig der aktiven Variante wird im diesem Kapitel erläutert.\\

\begin{figure}[h!]
  \begin{center}
    \includegraphics[scale=0.7]{5_1_klassenuebersicht.jpg}
  		  \caption{Einfache Übersicht der Klassen}
     \label{ttn.verbindung.klassen.loesung}
  \end{center}
\end{figure}


Um eine genauere Darstellung aller Klassen, bitte siehe Anhang \ref{A.PM.Diagramm}. Der Ablauf für das Speichern eines Parameters in einem Baumelement sieht folgendermaßen aus:
\begin{enumerate}
\item Benutzer verknüpft eine Anforderung an einem Baumelement
\item Der \textit{Listener} meldet das eine Änderung vorgenommen wird und gibt die Informationen weiter
\item Der \textit{VariantsManager} lädt und übergibt die nötige Informationen an das \textit{ParameterManager}
\item Der \textit{ParameterManager} lädt, sucht und erstellt ein \textit{ParameterTag}  und gibt das Befehl, dass ein \textit{ParameterTag} an ein Baumelement hinzugefügt werden muss.
\item Der \textit{ParameterManager} gibt das Befehl an das \textit{Listener} weiter
\item Das Befehl wird an die Befehlskette angehängt
\end{enumerate}

In den nächsten Unterkapiteln wird die Aufgaben der jeweiligen Klassen erläutert und der Ablauf ausführlicher erklärt.


%#######################################################################################
\subsection{VariantsManager}\label{sub.VariantsManager}
Die Hauptaufgabe des \textit{VariantsManager} ist die Varianten zu Zuordnen. Hier werden Baumelemente zu Varianten hinzugefügt und gelöscht. Dabei muss das Klassifikationsbaum neu gezeichnet werden. Diese Klasse kümmert sich auch, um die Umschaltung zwischen Varianten in der Variantenansicht (siehe Abbildung \ref{ttn.generic}) und die Richtige Darstellung der Baumelemente mit aktuelle Informationen.\\


Für die  Zwecke dieser Arbeit, wurde diese Klasse erweitert. Um die Aufgaben dieser Klasse zu erläutern, wird der Vorgang des Hinzufügen eines Parameters an einem Baumelement dargestellt. Der \textit{VariantsManager} setzt die Baumelement -ID und die Anforderungskennung in der Klasse \textit{ParameterManager}. Die Klasse dient als Schnittstelle zwischen der \textit{Listener} (siehe Kapitel \ref{sub:RSListner}) und die Klasse \textit{ParameterManager}  (siehe Kapitel \ref{sub:ParameterManager}).\\


Die Aufgabe des \textit{Listeners} lautet, dem \textit{VariantsManager} zu melden, wenn Änderung\footnote{Aktionen werden vom Benutzer ausgelöst anhand von Bedienelemente in der Benutzeroberfläche} geschehen werden oder geschehen sind. Im \textit{Listener} werden die benötigte Informationen gefiltert und an dieser Klasse weitergegeben, damit Änderung vorgenommen werden können. Im Fall der Parameterersetzung sind es die Identifikationsnummer eines Baumelementes und die Anforderungskennung der verlinkte Anforderung. Diese beide Kennungen diesen dazu, dass der \textit{VariantsManager} die jeweiligen Objekte (ein Baumelement und eine Anforderung) laden kann. Diese Objekte werden dann an der Klasse \textit{ParameterManager} übergeben.\\


Das angesprochene Baumelement muss aus das TESTONA-Diagramm geladen. Anhand der Identifikationsnummer werden die verschiedene Baumelemente einzeln abgefragt bis das richtige Baumelement gefunden wurde. Danach wird das Baumelement an das ParameterManager-Objekt übergeben.\\


Im Fall von der Anforderungskennung, werden mehrere Informationen übergeben. Die Anforderungskennung wird in einer lokalen Variable vom ParameterManager-Objekt gespeichert. Damit der Inhalt der Anforderung gelesen werden kann, müssen alle in TESTONA gespeicherte Anforderung an das ParameterManager-Objekt übergeben. Die gespeicherten \textit{Tags} in das TESTONA-Objekt werden abfragt um folgende Objekte an das ParameterManager-Objekt zu übergeben:


\begin{itemize}
\item \textbf{RequirementsConnetionTag}: beinhaltet die Information zur Datenbankverbindung, unter anderem die Verbindungsidentifikation (DOORS), das Modul(Tabelle) wo die Anforderung gespeichert ist und die Schnittstelle zur Datenbank. Anhand dieser Informationen wird die Verbindung zur Datenbank später aufgebaut.
\item \textbf{RequirementsListTag}: ist eine \textit{Map} mit alle in TESTONA gespeicherten Anforderungen. Das \textit{Key} ist die Anforderungsidentifikationsnummer und der Wert der Inhalt der Anforderung. Eine Anforderung aus dieser Liste wurde an das Baumelement verknüpft.
\item \textbf{VariantListTag}: eine Liste mit alle in TESTONA definierten Varianten.
\end{itemize}


Anhand dieser Informationen kann die Klasse \textit{ParameterManager} ein Parameter ein ein Baumelement hinzufügen. Parameter können nicht nur hinzugefügt werden, sondern aus gelöscht werden. Ein Parameter wird aus einem Baumelement gelöscht, indem die verknüpfte Anforderung vom Baumelement entfernt wird. Der Ablauft für das Löschen eines Parameter ist im Grunde der gleiche als beim hinzufügen. Zunächst werden die Funktionen des \textit{Listeners} erörtert.




%#######################################################################################
\subsection{ResourceSetListener}\label{sub:RSListner}
Das auslösendes Element ist, dass der Benutzer eine Anforderung mit einen Baumelement verknüpft. Um dieses Geschehen abzufragen muss ein \textit{Listener} programmiert werden, dass das Interface \textit{ResourceSetListener}\footnote{http://download.eclipse.org/modeling/emf/transaction/javadoc/1.0.3/org/eclipse/emf/transaction/ResourceSetListener.html} implementiert. Da dass \textit{VariantsManager} ein solcher \textit{Listener} bereits besitzt, wurde dieser erweitert. Die Aufgabe des \textit{ResourceSetListener} ist, dem Benutzer zu benachrichtigen, wenn sich eine Ressource ändert. Der \textit{Listener}  \glqq hört\grqq~ auf die reinkommenden Nachrichten (\textit{ResourceSetChangeEvent}) und wertet den Inhalt dieses Events aus.\\

Das \textit{Listener} implementiert zwei Methoden die für das Abfragen der Events relevant sind. Eine davon heißt \textit{resourceSetChanged(Event e)}. Diese Methode wird aufgerufen, wenn sich eine Ressource ändert (vom Element wo das \textit{Listener} angehängt ist). Am Anfang wurde diese Methode favorisiert für das abhören wenn eine Anforderung zu einem Baumelement verknüpft wird (um danach die Parameterersetzung zu triggern). Nach der Implementierung konnte ich feststellen, dass die Methode keine Schreibrechte auf die Baumelemente besitzt. Da die Ressource zu diesem Zeitpunkt sich schon geändert hat.\\

Um die Schreibrechte zu besitzen, wurde dann die Methode \textit{transactionAboutToCommit(Event e)} betrachtet. Diese Methode wird aufgerufen bevor eine Änderung geschieht. Es wird benachrichtigt, dass eine Änderung geschehen wird und welche Objekte (ein Baumelement und die angehängte Anforderung) betrachtet werden. Zu diesem Zeitpunkt besitzt die Klasse noch Schreibrechte auf die Baumelemente und kann Änderung vornehmen. Außerdem hat die Methode als Rückgabewert ein \textit{Command}. So können Kommandos an das Commandstack angekettet werden. Das Befehl, ein Parameter an ein Baumelement hinzufügen, wird zum richtigen Zeitpunkt ausgeführt, dank der Anordnung des Commandstacks. 


Die empfangene Nachricht beinhaltet folgende Elemente: 

\begin{itemize}
\item \textbf{Event: }ResourceSetChangeEvent, heißt die Ressourcen des Objektes haben sich geändert
\item \textbf{Notifier: } welches Objekt schickt die Nachricht
\item \textbf{Notification: }Beschreibung von der Änderung im \textit{Notifier}
\item \textbf{OldValue: } alter Wert, wenn nicht vorhanden \textit{null}
\item \textbf{NewValue: } neuer Wert, beim löschen von Werten \textit{null}
\end{itemize}


Hier wird als erstes der \textit{Notifier} abgefragt um auswerten zu können, ob die Nachricht für relevant ist. Es gibt drei Fälle das unterscheiden werden müssen. Eine Anforderung wird:


\begin{enumerate}
\item zum ersten Mal an einem Baumelement verknüpft
\item an einem Baumelement verknüpft, wo schon ändere Anforderungen verknüpft sind
\item gelöscht
\end{enumerate}


Für alle drei Fälle muss der \textit{Listener} wissen, an welches Element wurde das Event ausgelöst und welche Anforderung wurde hinzugefügt oder gelöscht. Zu bemerken ist, dass zu diesem Zeitpunkt noch nicht festgestellt werden kann, ob in der Anforderung ein Parameter definiert ist.\\


\subsubsection{Eine Anforderung hinzufügen}
Ist der \textit{Notifier} eine Instanz von \textit{TestonaClass}, so wurde eine Anforderung zum ersten mal an ein Baumelement verknüpft oder gelöscht. Um Unterscheiden zu können werden die Werte von \textit{NewValue} und \textit{OldValue} ausgewertet. Wenn \textit{NewValue} eine Instanz von \textit{RequirementTag} ist, dann wurde eine Anforderung hinzugefügt. Aus dem \textit{Notifier} wird die ID (repräsentiert das Baumelement) und aus der Variable \textit{NewValue} die Anforderungskennung (diese wurde in DOORS vergeben) abgefragt. Beide Werte werden an dem \textit{Variantsmanager} weitergegeben um danach der Befehl als Rückgabewert für das hinzufügen eines \textit{ParameterTags} zu bekommen.\\


Der Befehl wird nicht sofort als Rückgabewert der Methode \textit{transactionAboutToCommit} weitergegeben. Der Grund für diese Maßnahme heißt, dass es \textit{null} sein kann. Wenn kein Parameter in der Anforderung definiert ist, muss auch kein \textit{ParameterTag} an das Baumelement hinzugefügt werden und es muss kein Kommando ausgeführt werden. Weiterhin, in der Methode \textit{transactionAboutToCommit} werden andere \textit{Events} ausgewertet die auch Kommandos ausführen. Darum wird eine lokale Variable \texttt{command} definiert. Eine Eigenschaft der Klasse \textit{Command} ist, dass Kommandos aneinander angehängt werden können.

\begin{lstlisting}
Command command = null;
command = chain(command, manager.getParameter());
\end{lstlisting}

So wird das Kommando für das Hinzufügen eines \textit{ParameterTags} an der lokalen Variable \texttt{command} angehängt. Davor überprüft die Methode \textit{chain} ob einer der Parameter den Wert \textit{null} hat. Wenn keiner der Funktionsparameter den Wert \textit{null} hat, wird die Methode \texttt{chain(Command cmd)} der Klasse \textit{Command} aufgerufen.

\begin{lstlisting}
command.chain(CommandToAddParameter);
\end{lstlisting}

Der Vorgang für das Kommando wird für jedes zurückgegebenes Kommando durchgeführt, unabhängig vom Kommando (hinzufügen oder löschen). Am Ende der Methode kann das Kommando (wahrscheinlich bestehend aus mehrere Kommandos) als Rückgabewerte gegeben werden.\\


Wenn das Baumelement mindestens eine Anforderung besitzt und eine weitere Anforderung wird hinzugefügt, ist der \textit{Notifier} eine Instanz von \textit{RequirementTag} und \texttt{OldValue == null \&\& NewValue != null}. Aus dem \textit{Notifier} wird die ID des Baumelements abgerufen und \textit{NewValue} ist besitzt die Anforderungskennung. Bevor die neue Anforderung hinzugefügt wird, werden die alten gelöscht (wird danach erörtert) und alle neue hinzugefügt.\\


\subsubsection{Anforderungen löschen}
Ist der \textit{Notifier} eine Instanz von \textit{RequirementTag} so wurde eine Anforderung von einem Baumelement gelöscht oder es wurde eine neue Anforderung an einen Baumelement hinzugefügt, indem bereits Anforderung verknüpft sind. Wird eine Anforderung gelöscht, sind die Werte \texttt{OldValue != null \&\& NewValue == null} . In \textit{OldValue} befindet sich die Anforderungskennung die gelöscht werden soll. Der Löschvorgang teilt sich in zwei Fälle.\\


Der erste Fall beschreibt wenn eine Anforderung aus einem Baumelement gelöscht wird, dass nur eine Anforderung beinhaltet. Dieses \textit{Event} teilt sich in zwei Nachrichten. Die erste Nachricht beinhaltet die Anforderungskennung als ein \textit{String}. Die Anforderungskennung wird dann in einer lokalen Variable gespeichert, die mit der zweiten Nachricht ausgewertet wird. In der zweiten Nachricht ist der \textit{Notifier} eine Instanz von \textit{TestonaClass}. Es unterscheidet sich vom Hinzufügen einer Anforderung, weil die Variable \textit{OldValue} eine Instanz von \textit{RequirementTag} ist. Zur Sicherheit wird abgefragt ob die lokale Variable mit der Anforderungskennung ungleich \textit{null} ist. Danach kann aus dem \textit{Notifier} die ID des Bauelements abgefragt werden. Beide Werte werden an dem \textit{VariantsManager} übergeben um als Rückgabewert wird das Befehl für das Löschen eines \textit{ParameterTags}  erwartet.\\


Der zweite Fall beschreibt, dass eine oder mehrere Anforderung aus ein Baumelement gelöscht werden, wenn ein Baumelement mindestens eine Anforderung besitzt. Bevor die Anforderung hinzugefügt werden kann, werden die im Baumelement beinhaltende Anforderung zuerst gelöscht. Danach werden alle alte Anforderungen neu hinzugefügt sowohl als die neue Anforderung.\\


Wenn das Baumelement bereits eine Anforderung besitzt und eine zweite wird hinzugefügt, beinhaltet das \textit{Notifier} die ID des Baumelements und \textit{OldValue} die Anforderungskennung als \textit{String}. Besitzt das Baumelement mehr als eine Anforderung und eine neue wird hinzufügt, kann aus dem \textit{Notifier} die ID des Bauelements abgefragt werden. Der Unterschied liegt daran, dass \textit{OldValue} jetzt eine Liste von \textit{Stringwerte} ist. Jeder Wert in der Liste repräsentiert eine Anforderungskennung die gelöscht werden soll. So wird mit einer Schleife durch alle Elemente der Liste iteriert und jede Anforderung aus dem Baumelement gelöscht.\\



%#######################################################################################
\subsection{Parameter-\textit{Tag}}\label{sub.ParameterTag}
\subsubsection{ParameterTag}
Das Interface dient als Schnittstelle für den Zugriff auf das Inhalt von ParameterTagImpl. Das Interface erweitert \textit{Tag}. Das \textit{Tag} Interface ist ein in TESTONA allgemein implementiertes Modell, welches auf ein Interface und eine implementierende Klasse basiert. Für jedes \textit{Tag} (RequirementTag, VariantTag, etc.) gibt es ein eine Klasse. In das Interface wird immer das Tagtyp definiert um \textit{Tags} voneinander unterscheiden zu können.

\begin{lstlisting}[caption={ParameterTag Interface}, captionpos=b]
public static final String TAGTYPE = "ParameterTag";
\end{lstlisting}

Das Tagmodell wird benutzt um in TESTONA überall ähnliche Objekt zu haben, die verschiedene Funktionen erfühlen, aber die gleiche Richtlinie folgen

\subsubsection{ParameterTagImpl}
Die Klasse \textit{ParameterTagImpl} erweitert die Klasse \textit{TagImpl} und implementiert \textit{ParameterTag}. In dieser Klasse werden die Vorteile von das Tagmodell deutlicher. Die Klasse \textit{TagImpl} definiert sehr nützliche Methoden und Eigenschaften die in \textit{ParameterTagImpl} angewendet werden.\\

Jedes \textit{Tag} hat als Eigenschaft einen Namen. In diese Klasse entspricht der Name, ein Parametername der aus der Parametertabelle gelesen wurde. Weiterhin ist über \textit{ID} eine eindeutige Identifikationsnummer definiert, um \textit{Tags} des gleichen Typs voneinander zu unterscheiden.\\

Der Inhalt von das \textit{Tag} ist durch ein \textit{EcoreEMap<String, String>} definiert. Das erste String ist ein Schlüsselwert und das zweite String das Wert. Im diesem Fall ist der Schlüssel der Name einer Variante, und der Wert ist der Parameterwert des Parameters in dieser Variante.  Anhand der Methode \textit{getContent()} kann der Inhalt eines \textit{Tags} aufgerufen werden und mit der Methode \textit{addEntry()} werden Inhalte hinzugefügt. Das Gegenteil kann man mit der Methode \textit{deleteEntry()} erreichen. Wie die Methoden genauer angewendet werden, wird im Listing \ref{lst:CreateParamTag} gezeigt.

EINHEITEN\\

serialize und unserialize\\


%#######################################################################################
\subsection{ParameterManager}\label{sub:ParameterManager}

%\begin{figure}[h!]
%  \begin{center}
%    \includegraphics[scale=0.7]{.jpg}
%  		  \caption{Darstellung der Klasse ParameterManager in UML}
%     \label{ttn.ParameterMananger}
%  \end{center}
%\end{figure}

Die Klasse \textit{ParameterManager} beinhaltet eine private innere Klasse \textit{DoorsConnector}(siehe \ref{sub.DoorsConn}), diese kümmert sich um den Verbindungsaufbau mit DOORS.\\ Weiterhin baut die Klasse die vom \textit{Listener} weitergegeben Daten (Baumelement und Anforderungskennung) auf.\\



Anforderung eines Baumelementes beinhaltet Modul und Interface. wichtig für nächstes Kapitel\\

IndieModul : interId: doors, connId: localhost;36677, moduleId: /Simple Belt Warner 2/Simple Belt Warner Spec, moduleNbr: 0, id: doors,localhost;36677,/Simple Belt Warner 2/Simple Belt Warner Spec)\\


\begin{lstlisting}[caption={Erstellung der ParameterTag}, captionpos=b,label={lst:CreateParamTag}]
ParameterTag tempTag = new ParameterTagImpl();
		
for(int i = 0; i < attributesList.size(); i++){
	
	//An der nullte Stelle steht immer der Parametername
	if(i == 0) {
		tempTag.setName(req.getValue(attributesList.get(0)).getValueAsString());
	} else {
		//entferne " Value" vom Variantenamen 
		tempTag.addEntry(sAttrList.get(i).substring(0, sAttrList.get(i).lastIndexOf("Value")-1),
				req.getValue(attributesList.get(i)).getValueAsString());
	}
}
paramTagList.add(tempTag);
\end{lstlisting}


%#######################################################################################
\subsection{Kommandos}\label{sub.Command}
Für die Parameterersetzung wurden zwei Klassen implementiert, dass aus die Klasse \textit{RecordingCommand} erben. Die Klasse \textit{RecordingCommand} ist eine partielle Implementierung der Klasse \textit{Command}. Diese Klasse nimmt die  Manipulierung der Objekte der Subklasse auf und erzeugt daraus ein Kommando\footnote{14.02.2015; http://download.eclipse.org/modeling/emf/transaction/javadoc/1.2.3/org/eclipse/emf/transaction/RecordingCommand.html}. Eine der implementierten Klassen erzeugt ein Kommando für das Hinzufügen eines \textit{ParameterTags} an einem Baumelement, während der zweiten für das Löschen zuständig ist. Der Konstruktor der beiden Klassen haben die gleichen Parametern. Es werden der aktuelle Editor-Objekt übergeben, sowie das Baumelement ID und das \textit{ParameterTag}.\\


Die Methode \textit{doExecute()} wird überschrieben um die Änderung vorzunehmen. Als erstes werden alle Baumelemente geladen und durch iteriert bis das Baumelement gefunden wird, dass benötigt wird (anhand er ID). Soll ein \textit{ParameterTag} hinzugefügt werden, dann wird die Methode \textit{addTag(ParameterTag pTag)} aufgerufen.\\


Soll eine \textit{ParameterTag} gelöscht werden, dann wird die Methode \textit{removeTag(ParameterTag pTag)} aufgerufen. Danach muss auch der Name des Baumelementes neu gesetzt werden. Wenn ein Baumelement ein Parameter besitzt, je nach aktive Variante, wird die Darstellung geändert (siehe Kapitel \ref{sec.visialisierung}). Dafür wird der Name des Baumelementes geladen und falls eine Rücksetzung nötig ist, wird diese gemacht. Genaueres dazu wird in Kapitel \ref{sec.LoesungVisualisierung} veranschaulicht.\\

%#######################################################################################
\subsection{Die DOORS Verbindng}\label{sub.DoorsConn}
\subsubsection{DoorsConnector}
%\begin{figure}[h!]
%  \begin{center}
%    \includegraphics[scale=0.7]{.jpg}
%  		  \caption{Darstellung der innere Klasse DoorsConnector in UML}
%     \label{ttn.DoorsConnetor}
%  \end{center}
%\end{figure}

Die private innere Klasse \textit{DoorsConnector} baut die Verbindung zwischen den TESTONA und DOORS auf. Sie implementiert das Interface \textit{IConnectionListener}, dass ein \textit{Listener} für die Verbindung - Events umfasst. Für das Laden von Dateien aus DOORS benötigt die Klasse noch ein \textit{Listener} (IDataListener) und ein \textit{Adapter}(IReqLoadAdapter) \\

Als erstes wird von der Klasse \textit{ParameterManager} die Methode \textit{connectToDoors()} aufgerufen. Diese baut die Verbindung auf, indem gespeicherte Verbindungsdaten aufgerufen werden. Wie bereits in Kapitel \ref{sub.ParameterManager} erwähnt, beinhaltet eine Anforderung die Informationen wo die Anforderung gespeichert ist (welches DOORS Modul und über welches Interface das Modul zu erreichen ist). Für den Verbindungsaufbau werden folgende Objekte benötigt:

\begin{itemize}
\item \textbf{DataInterface: }Über diese Klasse erfolgt die Datenanfrage an DOORS. Die Verbindung wird aufgebaut sowie getrennt. Es werden als erstes die Ordner geladen, danach einzelne Projekte und die nötige Module. Es können verschiedene Darstellungen der Module auch geladen werden (diese müssen in DOORS definiert sein). Hier werden auch direkt einzelne Anforderungen angefragt. Relevant für diese Arbeit ist, dass hiermit das Modul Parametertabelle in DOORS geladen wird.

\item \textbf{PreferenceManagment: } Hier werden die in TESTONA gespeicherte Verbindungsdaten behandelt. Es können Microsoft Access Verbindungensdaten gespeichert werden, aber wir werde nur DOORS betrachten.

\item \textbf{Connector: }beschreibt eine einzelne Verbindung, hat ein \textit{DataIterface}- und \textit{PreferenceManagmentobjekt}

\item \textbf{ConnectionManager: }Singleton. Die Klasse handelt aktive und offene Verbindungen. Hier werden die \textit{ConnectionListeners} und das \textit{DataInterface} für den richtigen \textit{Connector} geregelt.

\end{itemize}


Um die Verbindung mit DOORS aufzubauen muss als erstes die Instanz des \textit{ConnecionManagers} lokal referenziert werden (weil es ein \textit{Singleton} ist, darf kein neues Objekt erzeugt werden). Wenn die Instanz des \textit{ConnectionManagers} geladen ist, kann jetzt der \texttt{connector} aus dem \textit{ConnectionManager} aufgerufen werden.

\begin{lstlisting}[caption={Verbindungsaufbau}, captionpos=b]
try {
	connector = ConnectorManager.getInstance()
				.getConnector(im.getInterId());
	dataInterface = conManager.getNewDataInterface(connector, this);
} catch (ExtensionException e) {
	e.printStackTrace();
}
\end{lstlisting}

 Um dem richtigen \texttt{connector} aufzurufen muss aus das \textit{IndieModul} (\texttt{im.getInterId()}, siehe Kapitel \ref{sub.VariantsManager}) die Interfacekennung als Übergabeparameter angegeben werden. Als nächstes kann über den \textit{ConnectionManager} eines neues \textit{DataInterface} erzeugt werden, wo der \texttt{connector} und das aktuelle Objekt (\textit{DoorsConnector}) übergeben werden.\\
 
Aus den in TESTONA gespeicherte Verbindungspräferenzen werden die Verbindungsparameter für DOORS geladen. An dieser Stelle braucht das \textit{DataInterface} die nötige \textit{Listeners} bevor die Verbindung aufgebaut wird. Durch die Methode \textit{addListener(listener)} wird das \textit{RequirementDataListener} und das \textit{IConnectionListener} (von \textit{DoorsConnetor} implementiert) gesetzt. Mit dem Aufruf der Methode \textit{connecteInterface(Verbindungsparameter)} wird das \textit{DataInterface} an DOORS verbunden.\\

Der Grund warum ein \textit{IConnectionListener} implementiert wird, lautet dass der Verbindungsaufbau in einem neuen Thread stattfindet. Das \textit{DoorsConnetor} Objekt wird über das \textit{IConnectionListener} benachrichtigt ob die Verbindung stattgefunden hat. Die Methode\texttt{interfaceConnected} (aus dem \textit{IConnectionListener}) wird aufgerufen, wenn die Verbindung erfolgreich entstanden ist.

\begin{lstlisting}[caption={Verbindungsaufbau war erfolgreich}, captionpos=b]
@Override
public void interfaceConnected() {
	connected = true;
	reqDataListener.setListener(reqLoadListener);
	dataInterface.loadModule(PARAM_PATH, this, false);
}
\end{lstlisting}

Der gesetzte \textit{RequirementDataListener} benötigt ein \textit{RequirementLoadListener}, dass eine rückmeldung gibt, wenn die Zeilen aus einer Tabelle fertig geladen worden sind. Die Tabelle wird anhand der Methode \texttt{loadModule} wird ein DOORS Modul (Tabelle) geladen. Welches Modul geladen wird, spezifiziert der Parameter \texttt{PARAM\_PATH}. Es gibt an wo sich das Modul in der DOORS Datenbank befindet. Der \textit{RequirementDataListener} erhält die Nachricht, dass ein Modul geladen worden ist. Weiter dazu wird im Kapitel \ref{sub.RequirementDataListener} erläutert.\\


Es wird angenommen, wenn der Benutzer die Parameterersetzung für ein Parameter wahrnimmt, dass er es für weitere Parameter wahrnehmen wird. Da die Datenmenge von einer Parametertabelle relativ gering ist, wurde hier, bezüglich der offene Frage bei dem Lösungsansatz (siehe Kapitel \ref{sec.parameterspeicherung}), die Parametertabelle komplett importiert. 
Ein weiterer Grund lautet, dass nicht für jeder Parameter erneut eine DOORS Verbindgun enstehen muss. So wird die Rechen- und Reaktionszeit von TESTONA so weit es geht gering gehalten. Die Parametertabelle wird erst bei der Verknüpfung von einer Anforderung mit einem Baumelement importiert und nicht beim Import der Anforderungen. 

Wenn die Parametertabelle komplett geladen wurde, ermöglicht die Methode \textit{closeConnection} die Verbindung mit DOORS zu beenden und das \textit{DataInterface} zu schließen.





\subsubsection{RequirementDataListener}\label{sub.RequirementDataListener}

Ist ein Event-Listener, dass auf die Rückmeldung vom Laden eines Moduls wartet. Relevant ist die Methode \texttt{onModuleLoad} die die Zeilen aus der Tabelle liest und speichert.\\

\begin{lstlisting}[caption={Laden der Parametertabelle nach Zeilen}, captionpos=b]
@Override
public void onModuleLoad(Module module, Object family, boolean reload){

	BasicRequirement baseReq;
	saveAttributesNames(module);
				
	for (String reqId : module.getRequirementIds()) {

		baseReq = dataInterface.getRequirement(module, reqId,
				reqLoadListener);

		if (baseReq.checkStatus(BasicRequirement.STATUS_LOADED)){
			paramReqList.add((Requirement) baseReq);
			fillParamTagList((Requirement) baseReq);
		}
	}
	dataInterface.flush(reqLoadListener);
}
\end{lstlisting}

Zu beachten ist, dass in DOORS jede Zeile in einer Tabelle als eine Anforderung (eng. Requirement) gesehen wird. Daher heißen Variablen und Methoden oft \glqq Requirement\grqq~ oder Abkürzung des Wortes (req). Die Methode \texttt{saveAttributesNames} speichert die im geladenes Modul vorhandenen Attribute. Im diesem Fall (treu zum Beispiel mit dem Auto) sind es:

\begin{itemize}\itemsep1pt
\item Paramenter Name
\item Default Value
\item Cabrio Value
\item Kombi Value
\item Limo Value
\end{itemize}

Diese Attribute repräsentieren die Varianten und ein Standardwert, sowie der Name des jeweiligen Parameter. In der Schleife werden alle Zeilen im Modul iteriert und geladen. Wenn die Zeile (\texttt{baseReq}) vollständig geladen wurde, wird diese in der globalen Liste \texttt{paramReqList} als ein \texttt{Requirement} Objekt gespeichert. Die Methode \texttt{fillParamTagList} erzeugt die \textit{ParameterTags} und wurde in Kapitel \ref{sub:ParameterManager} erläutert.\\

Wenn eine Zeile nicht vollständig geladen werden konnte, gibt es das \textit{RequirementLoadListener}, dass sich um das vollständige laden der Zeilen kümmert. Das \textit{RequirementLoadListener} wird in dieser Klasse instantiiert und von der \textit{DoorsConnetor} Klasse gesetzt. Weiter dazu im nächsten Kapitel.


\subsubsection{RequirementLoadListener}\label{sub.RequirementLoadListener}
Der \textit{RequirementLoadListener} reagiert wenn eine Zeile nicht völlstandigt geladen worden ist, und wartet bis diese geladen wird. Die Methode \texttt{onLoad} bekommt als Eingabeparameter eine Liste der nicht geladenen Zeilen.

\begin{lstlisting}[caption={Nachladen der Parametertabelle nach Zeilen}, captionpos=b]
public void onLoad(List<BasicRequirement> requirements) {

	for (BasicRequirement baseReq : requirements) {
		paramReqList.add((Requirement) baseReq);
		fillParamTagList((Requirement) baseReq);
		
	}
}
\end{lstlisting}

Diese Liste wird iteriert und wie in Kapitel \ref{sub.RequirementDataListener} in der globalen Liste \texttt{paramReqList} als ein \texttt{Requirement} Objekt gespeichert.\\

Weiterhin meldet diese Klasse wenn alle Zeilen aus das DOORS Modul geladen wurden. Das wartenden Dialogfenster aus \ref{sub:ParameterManager} wird benachrichtigt, dass es geschlossen werden kann. Die Benutzeroberfläche von TESTONA ist somit wieder für den Benutzer erreichbar.



%#######################################################################################
%#######################################################################################
\newpage
\section{Darstellung der Parameterwerte}\label{sec.LoesungVisualisierung}
\paragraph{}



%#######################################################################################
%#######################################################################################
\newpage
\section{Testfallgenerierung und Optimierung}
\paragraph{}
Erläuterung der Lösungen zu 4.3