\chapter{Fachliches Umfeld}\label{chp:fachlichesumfeld}

%#######################################################################################
%#######################################################################################
\section{TESTONA}\label{sec:Testona} 
\paragraph{}

Mit TESTONA wird dem Tester ein Tool angeboten, um signifikante Testszenarien und -umfänge strukturiert zu bestimmen. Mit dem Programm können komplette Testspezifikationen schnell und einfach generiert werden und überflüssige Testfälle vermieden werden. Bei Bedarf gibt es die Möglichkeit automatische generierte Testspezifikationen und Testfälle bequem mit Anforderungen durch Add-Ons an Standard-Werkzeugen zu verlinken (siehe \ref{sec:DOORS}). Somit werden robuste Schnittstellen für ein komfortables Anforderungs- und Testmanagement erzeugt. Ein sehr wichtiger Aspekt von TESTONA ist, dass es Branchen-unabhängig ist. Das heißt ein Tester kann TESTONA im jedem Fachgebiet benutzen, nicht nur bei Software- oder Funktionstests.\\

Mehr zu Testfällen und der Arbeitsweise dieses Programms werden in den nächsten Kapiteln erläutert, wie zum Beispiel die anerkannte Klassifikationsbaum-Methode.



%#######################################################################################
\subsection{Klassifikationsbaum-Methode}\label{ssec:KM}
\paragraph{}
1993 entwickelten K. Grimm und M. Grochtmann die Klassifikationsbaum-Methode zur Ermittlung funktionaler Blackbox-Tests im Bereich von eingebetteter Software. Die Methode wurde im Forschungslabor von Daimler-Benz in Berlin als Weiterentwicklung der Category-Partition Method (CPM) erforscht. Gegenüber CPM hat die Klassifikationsbaum-Methode eine graphische Baum-Darstellung und hierarchische Verfeinerungen für implizite Abhängigkeiten. Als Werkzeug wurde der \glqq Classification Tree Editor\grqq~ (CTE) \footnote{Entwickelt von Grochtmann und Wegener\cite{TestCaseDesign}. Bei Berner  \& Mattner aus rechtlichen Gründen zu TESTONA umbenannt} programmiert und unterstützt Partitionierung und Testfallgenerierung. Das Werkzeug von CPM konnte nur Testfälle generieren ohne Bestimmung der Testaspekte\cite{ClassificationTrees}.

 Diese Methode besteht aus zwei wichtigen Schritten:
\begin{itemize}
\item Bestimmung der Klassifikationen (testrelevante Aspekte) und Klassen (mögliche Ausprägungen).
\item Erzeugung von Testfällen aus Kombinationen von unterschiedlichen Klassen für alle Klassifikationen
\end{itemize}

Ansatzpunkt sind die funktionalen Anforderungen (siehe \ref{sec:DOORS}) eines zu testenden Objekts. Um die Testfälle zu definieren und zu erzeugen, folgt die Methode dem Prinzip des kombinatorischen Testentwurfs \cite{KlassifikationsbaumMethode}. Dieses Prinzip hilft bei der Detektierung von Fehlern in frühen Schritten des Testvorgangs. Nicht jeder einzelne Parameter steuert einen Fehler bei, eher werden Fehler durch die Interaktion verschiedener Parameter verursacht. Betrachten wir ein einfaches Beispiel. Ein Programm soll auf Windows oder Linux laufen, unter Verwendung eines AMD oder Intel Prozessors und mit Unterstützung des IPv4 oder IPv6 Protokolls. Das ergibt intuitiv acht verschiedene Testfälle ($2^{3} = 8$ Möglichkeiten, siehe Abbildung \ref{ttn.no_kombi}).\\

\begin{figure}[h]
  \begin{center}
    \includegraphics[scale=0.5]{os_cpu_proto_max_kombinatorik.png}
  		  \caption{Baum und Testfälle ohne Kombinatorik}
     \label{ttn.no_kombi}
  \end{center}
\end{figure}


Verwenden wir dafür den kombinatorischen Testentwurf \glqq paarweise Kombination\grqq~ (in Testona: \textit{pairwise(OS, CPU, Protokoll)} ), hätten wir nur vier Testfälle (siehe Tabelle \ref{table:4TestCases}). Durch diese Methode werden alle Kombinationspaare der Parameter mindestens durch einen Testfall abgedeckt\cite{CombinatorialSoftTesting}.\\

\begin{table}[h]


\begin{center}
	\begin{tabular}{|r||c|c|c|}
	 \hline
	 No. &OS &CPU &Protokoll\\
	 \hline \hline
	 1. &Linux &AMD &IPv6\\
	 \hline
	 4. &Linux &Intel &IPv4\\
	 \hline
	 5. &Windows &AMD &IPv6\\
	 \hline
	 8. &Windows &Intel &IPv4\\
	 \hline
	\end{tabular}
	
	\caption{Testfälle mittels paarweise Kombinatorik}
	\label{table:4TestCases}
\end{center}

\end{table}


Die Effizienz von diesem einfachen kombinatorischen Entwurf ist beim komplexeren System zu sehen. Hat ein System $20$ verschiedene Schalter und jeder Schalter $10$ verschiedene Einstellungen, so gibt es $10^{20}$ verschiedene Kombinationen. Durch Anwendung der paarweisen Kombination muss der Tester nur 180 Testfälle betrachten.\\

Verschiedene Experimenten haben gezeigt, dass durch die Verwendung von der paarweisen Kombinatorik die gleichen oder meistens mehrere Fehler entdeckt wurden, als mit der manuellen Testauswahl\footnote{Basierend auf funktionelle und technische Anforderungen, Use-Cases}. Paarweise Kombinatorik ist am meisten verbreitet, aber man kann durchaus auch die Drei-Wege-Kombinatorik verwenden. TESTONA implementiert standardmäßig Minimalabdeckung, Paarweise-, Drei-Wege- und N-Kombinatorik (wo N die maximale Anzahl an möglichen Parametern im Klassifikationsbaum ist, auch vollständige Kombinatorik genannt)\cite{CombinatorialSoftTesting}.\\




%#######################################################################################
\subsection{Testfälle und Testfallgenerierung}
\paragraph{}

Unter einem Testfall versteht man, die Beschreibung eines elementaren Zustands eines Testobjekts. Hierfür werden Eingangsdaten benötigt (Parameterwerte, Vorbedingungen) und ein erwarteter Folgezustand. Mit TESTONA werden Testfälle für eine vereinbarte Testspezifikation definiert. Laut IEEE 829 ist unter Testspezifikation zu verstehen: die Durchführung von

\begin{itemize}
\item\textbf{ Testentwurfsspezifikation}: verfeinerte Beschreibung der Vorgehensweise für das Testen einer Software
\item \textbf{Testfallspezifikation}: dokumentiert die zu benutzenden Eingabewerte und erwarteten Ausgabewerte
\item \textbf{Testablaufspezifikation}: Beschreibung aller Schritte zur Durchführung der spezifizierten Testfälle.
\end{itemize}

Da TESTONA in allen Testphasen einsetzbar ist, kann die Arbeitszeit effizient reduziert werden. Dazu hilft auch die automatische Testfallgenerierung und die verschiedene kombinatorischen Möglichkeiten (siehe \ref{ssec:KM}) . Somit kann der Tester einen besseren Zeitplan erzeugen und die Arbeitskräfte zielbewusst an der Ausführung und Auswertung der Testfälle beschäftigen.\\

Allgemein wird ein Test laut des ISTQB-Glossar\footnote{International Software Testing Qualification Board} folgendermaßen definiert:

\begin{center}
\textit{
Der Prozess, der aus allen Aktivitäten des Lebenszyklus besteht (sowohl statisch als auch dynamisch), die sich mit der Planung, Vorbereitung und Bewertung einer Softwareprodukts und dazugehörige Arbeitsergebnisse befassen. Ziel des Prozesses ist sicherzustellen, dass diese allen festgelegten Anforderungen genügen, dass sie ihren Zweck erfüllen und etwaige Fehlerzustände zu finden.}\cite{SoftwareTestEmbSys}\\

\end{center}

Anhand der generierten Testfälle und die Benutzung von kombinatorischen Möglichkeiten kann der Tester einfach eine präzise Testtiefe erreichen. Die Testtiefe wird anhand der Durchführung einer Risikoanalyse und der Auswertung der Kritikalitätsstufe (sehr hoch, hoch, mittel und tief) des Systems vereinbart. Das soll heißen, dass die Testtiefe für ein Flugzeug (Kritikalitätsstufe = sehr hoch\footnote{Das Fehlverhalten kann zu Verlust von Menschenleben führen, die Existen des Unternehmens gefährden}) viel höher und genauer ist, als die Kompatibilität eines Bildschirmes mit ein Graphiktreiber (Kritikalitätsstufe = tief\footnote{Das Fehlverhalten kann zu geringen materiellen oder immateriellen Schäden führen}).\\

Anhand der Risikoanalyse und der Kritikalitätsstufe wird auch eine vereinbarte Testabdeckung und ein Testfallermittlungsverfahren erreicht. Im Falle des Flugzeugs wird eine Kombination von White und Blackbox-Methoden mit sehr hoher Testtiefe ausgeführt. Dagegen, im Falle des Bildschirmes, kann intuitives Testen mit geringer Testtiefe angewendet werden.\cite{ApplicationEngineering}

%#######################################################################################
%#######################################################################################
\subsection{Abhängigskeitsregeln}
\paragraph{}

Abhängigkeitsregeln werden vereinbart um überflüssige Testfälle zu vermeiden, bzw. um Vorbedingungen für bestimmte Testszenarien festzulegen. Abhängigkeitsregeln werden Mithilfe von boolische Algebra definiert wie folgende Abbildung \ref{ttn.depencyRules}zeigt:

\begin{figure}[h!]
  \begin{center}
    \includegraphics[scale=0.5]{dependency_rules_2.png}
  		  \caption{Abhängigkeitsregeln Bearbeitung}
     \label{ttn.depencyRulesEdit}
  \end{center}
\end{figure}

Boolesche Operatoren:\\

\begin{tabular}{ll}
AND &: Konjunktion\\
NAND &: negierte Konjunktion\\
OR &: Disjunktion\\
NOR &: negierte Disjunktion\\
XOR &: ausschließende Disjunktion\\
\% &: \glqq don't care\grqq~ Operator\\
=> &: vom A folgt B\\
<=> &: A ist gleichwertig wie B\\
NOT &: Negation\\
\end{tabular}


Durch die Verwendung dieser Operatoren wurde folgende Abhängigkeitsregel definiert:

\begin{figure}[h]
  \begin{center}
    \includegraphics[scale=0.5]{dependency_rules.png}
  		  \caption{Abhängigkeitsregeln Übersicht}
     \label{ttn.depencyRules}
  \end{center}
\end{figure}

Nehmen wir die Regeln wahr, so will der Tester die Klasse \glqq IPv4\grqq~ für diese Tests nicht betrachten. Also werden nur Testfälle erzeugt, in denen dieser Parameter nicht vorkommt.


\begin{figure}[h]
  \begin{center}
    \includegraphics[scale=0.5]{dependency_rules_noParam.png}
  		  \caption{Testfälle mit Anwendung der Abhängigkeitsregeln aus \ref{ttn.depencyRulesEdit} und \ref{ttn.depencyRules}}
     \label{ttn.testsequence}
  \end{center}
\end{figure}

In diesem einfachen Beispiel wird nur ein Parameter ausgeschlossen, aber durch die Abhängigkeitsregeln kann der Tester durchaus komplexere Fälle einleiten. Betrachten wir folgendes Szenario: Ein Licht wird mit einer 5V Spannung durch eine Batterie versorgt. Die Spannung der Batterie befindet sich momentan im niedrigen Bereich. Das System hat zwei Lichtschaltern um das Licht zu steuern. Es soll die Reaktion des Systems getestet werden, wenn eins der beiden Lichtschaltern betätigt wird. Um so ein Fall zu prüfen, werden zwei verschiedene Regeln angelegt:

\begin{enumerate}
\item \textit{Lichtschalter1 XOR Lichtschalter2}
\item \textit{Spannung NOT 5V}
\end{enumerate}

Die erste Regel besagt, dass mindestens einer der beiden Lichtschalter betätigt werden muss, um das Licht einzuschalten. Die zweite Regel spezifiziert, dass nur der Fall betrachtet wird, wenn die Spannung der Batterie unter die 5V Grenze liegt. Daraus werden nur Testfälle erzeugt, die diese beide Kriterien erfüllen.





%#######################################################################################
%#######################################################################################
\newpage
\section{IBM Rational DOORS}\label{sec:DOORS}
\paragraph{}

Quality Systems \& Software (QSS) hat Anfang der 90er Jahre DOORS (Dynamic Object Oriented Requirements System) entwickelt. Die Firma Telelogic kaufte im Jahr 2000 QSS, die wiederum 2008 von IBM übernommen wurde. DOORS ist eine Anforderungsmanagement Software und ermöglicht die Verwaltung und strukturierte Aufzeichnung von Anforderungen (als Objekte). Durch eine tabellarische Ansicht der Anforderungen können die Anforderungen und die zugehörige Eingenschaften abgelesen werden. Als Eingenschaften sind eine eindeutige Identifikationsnummer, sowie vom Benutzer ausgewählte Attribute.\\


\begin{figure}[h]
  \begin{center}
    \includegraphics[scale=0.5]{doorsTable.png}
  		  \caption{Beispiel einer Tabelle in DOORS}
     \label{doors.bespiel}
  \end{center}
\end{figure}


In DOORS wird eine Tabelle \glqq Modul\grqq~ genannt. Jede Zeile innerhalb eines Moduls  ist ein Objekt und die Spalten für jedes Objekt bezeichnet man als Attribut. Ein Objekt kann Unterobjekte besitzen, indem Alternativen und weitere Anforderungen beschrieben werden (siehe Abbildung \ref{doors.bespiel}).


Um Anforderungen im Laufe des Projektes zu verfolgen (Tracing), können Anforderungen miteinander verlinkt werden. DOORS basiert sich auf eine Client - Server Anwendung mit einer proprietären Datenbank. \\


Es werden auch Schnittstellen für den Datenaustausch zur Verfügung gestellt (Testmanagement -, Modellierungs - und Changemanagementwerkzeuge) dank der Unterstützung von RIF (Requirements Interchange Format). Durch die Skriptsprache \glqq DXL\grqq~ (DOORS eXtention Language) erhält TESTONA Zugriff auf die gespeicherten Module in DOORS. \cite{Doors} \cite{Anforderungsmanagement}\\

TESTONA besitzt Klassen die DXL-Scripte ausführen, damit der Entwickler einfach und sicher Daten in DOORS abrufen kann.


%#######################################################################################
%#######################################################################################
\newpage
\section{Variantenmanagement}\label{sec:VarManag}
\paragraph{}

Variantenmanagement, wie das Wort schon verrät, behandelt verschiedene Varianten eines Produktes. Um eine Variante besser zu definieren, betrachten wir folgendes Beispiel: ein Auto \glqq A\grqq~ soll in verschiedenen Modellen gebaut werden: Kombi, Limousine und Cabrio. Alle Modellen von \glqq A\grqq~ haben die Eigenschaften eines Wagens (4 Räder, Personenkraftwagen, Türen, etc), aber sie unterscheiden sich untereinander durch die Anzahl der Passagiere oder der Größe des Wagens. Daher kann jedes gebaute Modell von \glqq A\grqq~ als eine Variante betrachtet werden.\\


Anhand des Beispieles wird klar, dass durch die steigenden Produktkomplexität bzw. -vielfalt die Identifikationsmerkmale zur Definition einer Produktvariante immer schwieriger zu vereinbaren sind. Für diese Masterarbeit ist eher wichtig, Identifikationsmerkmale zu definieren, die dazu führen, dass die Tests oder der Testablauf eines Produktes geändert werden muss (mehrere Testfälle sind nötig, neue Parameterwerte). Das heißt, wenn ein Auto als Kombi gebaut wird und danach als Cabrio angeboten wird, müssen (unter anderem) die ganze Dachfunktionen geprüft werden. Das führt dazu neue Testspezifikationen, Testabläufe und Testfälle zu erstellen, die die Funktionalität des Daches überprüfen.\\


Varianten werden oft mit Features verwechselt. Der Unterschied zwischen ein Feature und einer Variante ist sehr fein und oft abhängig vom Betrachtungspunkt. Um dem Unterschied deutlicher zu machen, werden Änderung in der Funktionalität oder Konstruktion des Autos als einer Variante betrachtet. So sollte man die Karosseriefarbe als ein Feature betrachten, weil theoretisch eine Änderung der Farbe keine Änderung in der Funktionalität oder sich auf die Leistung des Autos auswirkt (somit werden die Tests oder Testabläufe nicht beeinflusst). Ein Auto mit verschiedener Lackierung des selben Modells erzeugt keine Änderung im Testaufwand.\cite{VarMan2}\\


\begin{figure}[h]
  \begin{center}
    \includegraphics[scale=0.4]{varianten.png}
  		  \caption{Betrachtung von Varianten eines Wagens von der generischen Testspezifikation zur variantenspezifische Testspezifikation aus \cite{VarMan1}}
     \label{variantsOverview}
  \end{center}
\end{figure}


Das Variantenmanagement wird im Allgemeinen dazu benutzt um bei komplexere Produkte einen besseren Überblick zu haben. Hinsichtlich das Testen, können Änderung des Produktes besser erkannt werden. Somit werden die Testfälle und Testabläufe an der jeweiligen Variante angepasst. Die Abbbildung \ref{variantsOverview} betrachtet die Länge jedes Modell und erstellt aus die generische Spezifikation einzelnen Spezifikationen für jede Variante. Die Testfälle werden mit den richtigen Werten erstellt und eine bessere Testabdeckung kann gewährleistet werden.\\


In TESTONA werden  die in DOORS definierten Produktvarianten importiert. Baumelement werden an der jeweiligen Variante hinzugefügt. Wenn der Benutzer eine Variante auswählt, werden nur die in dieser Variante gültige Baumelemente angezeigt. So wird die Übersicht verbessert und der Tester kann Testfälle genauer erstellen.




%#######################################################################################
%#######################################################################################
\newpage
\section{Entwicklungsumgebung und Programmiersprache}
\paragraph{}
TESTONA wurde ursprünglich in der Programmiersprache \glqq Pascal\grqq entwickelt. Mit der Weiterentwicklung wurde das Programm an \glqq C\grqq portiert. Durch den Kauf von Berner \& Mattner in 2008 wurde TESTONA bis zum jetzigen Zeitpunkt auf Java übersetzt und wird seit 2010 mit der Entwicklungsumgebung Eclipse entwickelt.

%#######################################################################################
\subsection{Eclipse}
\paragraph{}
Eclipse ist der Nachfolger von \glqq IMB Visual Age for Java\grqq~ und ist ein quelloffenes Programmierwerkzeug zur Entwicklung verschiedener Arten von Software. Ursprünglich wurde Eclipse als integrierte Entwicklungsumgebung für Java benutzt. Dank seiner Bedienbarkeit und Erweiterung ist Eclipse mittlerweile für die Entwicklung in verschiedenen Programmiersprachen bekannt (unter anderem C/C++ und PHP). Die am 25 Juni 2014 veröffentliche Version \glqq Luna\grqq~ (Eclipse 4.4) ist der aktuellste Stand der Software. \\

Mit Eclipse 3.0 hat sich die Grundarchitektur von Eclipse geändert. Seit diesem Zeitpunkt ist Eclipse nur ein Kern, der einzelne Plug-ins lädt. Jedes Plug-in stellt eine oder verschiedene Funktionalitäten zur Verfügung. Darauf aufbauend existiert die \glqq Rich Client Platform\grqq~ (RCP). Diese ermöglicht Entwicklern Anwendungen zu programmieren, die auf das Eclipse Framework aufbauen, aber unabhängig von der Eclipse IDE ist\cite{EclipseRCP} \cite{Eclipse}. Dies ist einer der Hauptgründe warum TESTONA zu Java und Eclipse migriert wurde. Durch RCP können verschiedene Programmversionen (Light, Express, Professional, Enterprise) besser verwaltet werden. Durch die RCP Architektur werden auch verschiedene Funktionalitäten voneinander getrennt. Das hilft bei der Weiterentwicklung sowie bei der Pflege des Programms, da mehrere Entwickler gleichzeitig an verschiedenen Plug-ins arbeiten können. Der Aufwand wird niedrig gehalten, weil nur ein Workspace genutzt wird. Bei der Erstellung von verschiedene Versionen werden nur die Plug-ins geladen, die nötig sind.


%#######################################################################################
\subsection{Plug-ins}
\paragraph{}
Ein Plug-in ist die kleinste ausführbare Softwarekomponente für Eclipse. Um eine Anwendung mit Eclipse RCP zu schreiben, werden mindestens diese drei Plug-ins benötigt:

\begin{itemize}
\item Eclipse Core Plattform: steuert den Lebenszyklus der Eclipse Anwendung
\item Stardard Widget Toolkit: Programmierbibliothek zur Erstellung von grafischen Oberflächen
\item JFace: User Interface Toolkit für komplexere Widgets
\end{itemize}

Weitere Plug-Ins, von der Eclipse Foundation implementiert, stehen den Programmierern zur Verfügung und unter \textit{marketplace.eclipse.org} können auch von Privatentwicklern programmierte Plug-Ins heruntergeladen werden. \cite{Eclipse}\\

Plug-ins beinhalten den Java-Code der, wie gewöhnlich, in verschiedene Pakete und Klassen strukturiert werden kann. Sinnvoll ist, dass jedes Plug-in eine Funktionalität des gesamten Programms repräsentiert, wie zum Beispiel, Variantenmanagement oder Autosave.\\

Testona besteht momentan aus viele verschiedene Plug-ins, wobei jedes Plug-in eine bestimmte Funktion oder Feature des Programms implementiert. Zum Beispiel gibt es für jede Version (Light, Express, Professional und Enterprise) von TESTONA ein Plug-in indem die nötige Plug-ins für die gewünschte Version geladen werden.\\

Ein Plug-in besteht in der Regel aus folgenden Einheiten:

\begin{itemize}
\item \textbf{JRE System Library:} beinhaltet alle Systembibliotheken von der Java Runtime Enviroment, dass der jeweilige Plug-in benötigt
\item \textbf{Plug-in Dependencies:} schließt die Abhängigkeiten des Plug-ins mit der Eclipse Umgebung und anderen implementierten Plug-ins ein
\item \textbf{src:} bezieht die Pakete und Java-Klassen ein
\item \textbf{icons:} hier befinden sich die Bilderdateien (.gif, .png, etc) die in den Klassen für die Benutzeroberfläche  aufgerufen werden
\item \textbf{META-INF:} gibt eine Übersicht aller Einstellungen des Plug-ins, sowie die Möglichkeit diese über eine graphische Oberfläche zu bearbeiten
\item \textbf{build.properties:} beinhalten die Einstellungen für das Compilieren des Plug-in. Es kann auch über die META-INF Datei bearbeitet werden.
\item \textbf{plugin.xml:} hier werden die nötigen Erweiterungen für das Plug-in definiert. Es ist möglich direkt die \textit{xml} Datei zu bearbeiten, oder die META-INF Oberfläche benutzen.
\end{itemize}

Ein Plug-in kann durchaus mehrere Einheiten oder Elemente beinhalten, wie zum Beispiel weitere \textit{resource} Ordner oder ein Dokumentationsordner mit wichtigen Dokumenten zum Plug-in.




%#######################################################################################
\subsection{Standard Widget Toolkit (SWT)}
\paragraph{}
SWT ist eine IBM Programmierbibliothek (seit 2001) für die Programmierung grafischen Oberflächen unter Java. Die Bibliothek benutzt, im Gegensatz zu Swing\footnote{Programmierschnittstelle und Grafikbibliothek für Java zum programmieren von grafischen Benutzeroberflächen.}, die nativen grafischen Elemente des jeweiligen Betriebssystems und ermöglicht die Erstellung von Anwendungen, die optisch ähnlich wie die nativen Anwendungen des Betriebssystems aussehen. Durch die Verwendung der nativen grafischen Elemente, kann das Toolkit sofort Änderungen in das \glqq look and feel\grqq~ des Betriebssystems in der Anwendung aktualisieren und beinhaltet ein konstantes Programmiermodel in alle Plattformen.
\cite{EclipseSWT}\\

SWT beinhaltet sehr viele komplexe Eigenschaften. Aber für eine robuste und benutzbare Anwendung zur Programmieren sind nur die Grundkenntnisse nötig. Eine typische SWT Anwendung hat folgende Struktur:

\begin{itemize}
\item Ein \textit{Display} deklarieren (repräsentiert die SWT Modus)
\item Erstellen eines \textit{Shell}, welche als Hauptfenster dient
\item Erzeugung eines Widgets
\item Initialisierung der Widgetsparameter
\item Öffnen des Fensters
\item Starten der Event-Schleife bis eine Abbruchbedingung erfüllt wird (Schließen des Fenster vom Benutzer)
\item Entsorgen des Displays
\end{itemize}

\begin{lstlisting}[caption={Beispiel einer SWT Anwendung}, captionpos=b]
   public static void main (String[] args) {
      Display display = new Display();
      Shell shell = new Shell(display);
      Label label = new Label(shell, SWT.CENTER);
      label.setText("Hello_world");
      label.setBounds(shell.getClientArea());
      shell.open();
      while (!shell.isDisposed()) {
         if (!display.readAndDispatch())
         		display.sleep();
      }
      display.dispose();
   }
\end{lstlisting}


SWT wird in TESTONA eingesetzt, damit das Aussehen der Benutzeroberfläche automatisch aktualisiert an des jeweilige Betriebssystem angepasst wird. Es erspart auch sehr viel Entwicklungszeit, da die Grafikkomponenten des Programms nicht für jedes Betriebssystem programmiert werden müssen.





%#######################################################################################
\subsection{JFace}
\paragraph{}
JFace ist ein User Interface Toolkit, das auf die von SWT gelieferten Basiskomponenten setzt und stellt die Abstraktionsschicht für den Zugriff auf die Komponenten bereit. Es beinhaltet Klassen zur Handhabung gemeinsame Programmieraufgaben, wie zum Beispiel:


\begin{itemize}
\item Viewers: Verbindung von Oberflächenelementen zum Datenmodell
\item Actions: definiert Benutzeraktionen und spezifiziert wo diese zur Verfügung stehen
\item Bilder und Fonts: gemeinsame Muster für den Umgang mit Bilder und Fonts
\item Dialoge und Wizards: Framework für komplexere Interaktionen mit dem Benutzer
\item Feldassistent: Klassen die dem Benutzer für richtige Inhaltsauswahl bei Dialogen oder Formularen Hilfe anbieten
\end{itemize}


SWT ist komplett unabhängig von JFace (und Plattform Code), aber JFace wurde konzipiert um SWT  bei allgemeinen Benutzerinteraktionen zu unterstützen. Eclipse ist wohl das bekannteste Programm das JFace benutzt.\cite{EclipseHelp}\\



%#######################################################################################
\subsection{Tags}\label{sub.Tags}
\paragraph{}

Ein \textit{Tag} ist ein Objekt, das auf EMF (Eclipse Modeling Framework) basiert. EMF ist ein Java Framework der Eclipse-Open-Source Gemeinschaft für die automatische Erzeugung von Quelltext aus Modellen \cite{EclipseRCP}.  Bei EMF wird die Problemstellung in einem Klassendiagramm beschrieben. Hier werden die Eigenschaften und Attribute der Klassen eingetragen um daraus wird der Quelltext generiert. Die Objekte können ohne großen Aufwand serialisiert und validiert werden und mögliche Fehlerquellen werden ausgeschlossen. In das Ecore-Modell werden die Klassen definiert und in Pakete eingeteilt. Der erzeugte Quelltext kann dann vom Benutzer ergänzt werden.\\



\textit{Tags} dienen hauptsächlich dazu um Eigenschaften anderer Objekte zu beschreiben. In \textit{VariantsTag} werden zum Beispiel  Varianten gespeichert. Wenn in einem Projekt verschiedene Varianten zur Verfügung stehen, wird es dazu kommen, dass ein Baumelement nicht in allen Varianten auftritt. Das Baumelement wird ein \textit{VariantsTag} besitzen, das spezifiziert zur welchen Varianten das Baumelement gehört. in \textit{RequirementTags} werden die Anforderungen gespeichert die an einem Baumelement verknüpft worden sind. Für die farbige Trennung zwischen Knotenpunkten im Klassifikationsbaumeditor ist ein \textit{ColoringTag} zuständig.\\

Die Grundstruktur eines \textit{Tags} besteht aus:

\begin{itemize}
\item Name
\item Identifikationsnummer
\item Typ
\item Inhalt
\item Zugriffsrechte
\end{itemize}


Je nach Zweck können die Klassen um weitere Attribute erweitert werden. Durch die Vererbung und Quelltextgenerierung wird sichergestellt, dass alle \textit{Tags} innerhalb TESTONA kompatibel sind. Anhand der Serialisierung können die \textit{Tags} in die TESTONA Datei im XML-Format gespeichert werden und auch ausgelesen werden. Um Parameter in Baumelementen zur speichern, wird auf diesem Modell aufgebaut. Weitere Erörtungen in Kapitel \ref{sec.parameterspeicherung} und \ref{sub.ParameterTag} .