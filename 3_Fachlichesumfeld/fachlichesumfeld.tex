\chapter{Fachliches Umfeld}\label{chp:fachlichesumfeld}
Die Quellen dieses Kapitel sind aus ......

%#######################################################################################
%#######################################################################################
\section{TESTONA}
\paragraph{}
%was ist testona
%bäume
%eclipse basierend

%#######################################################################################
\subsection{Klassifikationsbaum-Methode}
\paragraph{}

%#######################################################################################
\subsection{Testfälle und Testfallgenerierung}
\paragraph{}



%#######################################################################################
\subsection{Abhängigsleitsregeln}
\paragraph{}



%#######################################################################################
\subsection{Variantenmanagement und IBM Rational DOORS}
\paragraph{}
%kopplung zwischen testona und doors, parameter der anforderungen



%#######################################################################################
%#######################################################################################
\section{Entwicklungsumgebung und Programmiersprache}
\paragraph{}

%#######################################################################################
\subsection{Eclipse}
\paragraph{}


%#######################################################################################
\subsection{Plugins}
\paragraph{}


%#######################################################################################
\subsection{Java}
\paragraph{}


%#######################################################################################
\subsection{Java SWT}
\paragraph{}