\chapter{Zusammenfassung und Ausblick}\label{chp:zusammenfassung}
%Lehnen Sie sich zurück von Ihrem Terminal und versuchen ein wenig Abstand zu den vielen Detail-Problemen Ihrer Diplomarbeit zu gewinnen:
%Was war wirklich wichtig bei der Arbeit? 
%Wie sieht das Ergebnis aus?
%Wie schätzen Sie das Ergebnis ein?
%Gab es Randbedingungen, Ereignisse, die die Arbeit wesentlich beeinflußt haben?
%Gibt es noch offene Probleme?
%Wie könnten diese vermutlich gelöst werden?



%#######################################################################################
%#######################################################################################
\newpage
\section{Zusammenfassung}
\paragraph{}
Mit dieser Masterarbeit sollte die Erweiterung und Verbesserung des Berner \& Mattner Werkzeuges TESTONA erreicht werden. Dabei soll die Testfallgenerierung optimiert werden. Nachdem Anforderungen und Varianten aus DOORS Module importiert wurden, können in DOORS definierte Parameterwerte ebenso importiert werden. Die Parameterwerte können dynamisch in Baumelementen dargestellt werden, immer in Abhängigkeit von der aktiven Produktvariante. Dabei wird der Tester unterstützt, indem eine bessere Übersicht erreicht wird. So kann eine höchstmögliche Testabdeckung einfacher erzielt werden.\\


Durch das Speichern und Darstellen von importierten Parameterwerten aus DOORS wurde die Testfallgenerierung in TESTONA verbessert. Mit dem automatischen Import der Parameterwerte wurde dem Benutzer viel Arbeit abgenommen. Dabei erhält der Benutzer eine bessere Übersicht der erzeugten Testfällen und kann diese den reellen Werten gegenüberstellen. Dies führt zu Zeitersparnis und niedrigen Testkosten.\\


Anhand der vorhandenen Parameterwerte können auch gezielt spezielle Testfälle erzeugt werden. Ist zum Beispiel ein Teil eines Produktes sehr komplex, kann der Benutzer anhand der vorhandenen Parameterwerte speziell für dieses Produkt die notwendigen Testfälle generieren.\\


Weiterhin besteht die Möglichkeit, doppelte Testfälle leichter zu erkennen und zu vermeiden. Diese entstehen durch gleiche Parameterwerte in verschiedenen Baumelementen mit dem gleichen Oberknoten in einer aktiven Variante.\\


%#######################################################################################
%#######################################################################################
\newpage
\section{Persönliches Fazit}
\paragraph{}
Durch Verwendung der Programmiersprache Java konnte ich meine Kenntnisse und Erfahrungen in dieser Sprache bereichern. Anhand der Arbeit in einem umfangreichen Programm wie TESTONA habe ich gelernt, Teil eines größeren Entwicklungsteams zu sein. Der Ablauf von Teamabsprachen, wöchentlichen Statusmeetings sowie Vereinbarungen sind mir vertrauter geworden.\\


Da TESTONA auf RCP basiert, konnte ich viele neue Aspekte von Java und von Programm\-architekturen lernen. Diese Masterarbeit befasst Themen wie Testen, Qualitätssicherung und Softwareentwicklung, die meiner Interessen sehr nahe liegen. Die erworbenen Kenntnissen werde ich in meiner zukünftigen Berufsleben erfolgreich anwenden können.\\


Da während der Arbeit meine Betreuer aus der Firma Berner \& Mattner sich geändert haben, wurden sich auch die Anforderungen etwas geändert. Die Änderung der Betreuer hat auch dazu beigetragen, dass organisatorisch nicht alles optimal lief. Ich schätze das Ergebnis als durchaus positiv und bereichernd für TESTONA. Ich glaube es wäre auch sehr interessant gewesen, die Abhängigkeitsregeln für die Testfallgenerierung genauer zu betrachten.