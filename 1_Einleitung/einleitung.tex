\chapter{Einleitung}\label{chp:einleitung}
\paragraph{}
%Gefordert ist......
Ziel dieser Masterarbeit ist die Erweiterung und Verbesserung des Berner \& Mattner Werkzeuges TESTONA. Dieses Programm bietet Testern ein Werkzeug für eine strukturierte und systematische Ermittlung von Testszenarien und -umfänge \cite{TESTONA}. Im Kapitel \ref{sec:Testona} wird weiteres zu dieses Programm und die Funktionsweise erläutert.\\

Die Erweiterung des Programmes besteht aus verschiedene Themen. Eins davon behandelt die Testfallgenerierung und die jeweilige Testabdeckung. Hier soll garantiert werden, dass bei einer automatischen Testfallgenerierung, eine höchstmögliche Testabdeckung erzielt wird.\\

Die Testfallgenerierung wird in dieser Arbeit beeinflusst, indem stärker die Produktvarianten betrachten werden. Verschiedene Varianten beinhalten verschiedene Parameter und Produktkomponenten. Die Parameterwerte definieren auch verschiedene Produktvarianten. Durch das Add-On MERAN für die Anforderungsmanagementsoftware "'IBM Rational DOORS"` können Anforderungen direkt in  TESTONA importiert werden. Dabei sollen automatisch die Parameterwerte zur der jeweilige Produktvariante zugeordnet werden. Aus diesem Grund kann es zu Konflikte bei der Testfallgenerierung kommen, bzw. inkohärente Testfälle.\\

Um solche Probleme zu vermeiden oder umgehen, gibt TESTONA den Testern die Möglichkeit Abhängigkeitsregeln anzulegen. Hier können Anfangsbedienungen sowie Sonderbedienungen definiert werden. Dabei muss wiederum geachtet werden, dass die Produktvarianten nicht verletzt werden. Weiteres zu den Themen und Begriffen wird im Kapitel \ref{chp:fachlichesumfeld} verdeutlicht.\\

Im Kapitel \ref{chp:aufgabenstellung} wird genauer die Aufgabe dieser Masterarbeit erläutert und in den Kapiteln \ref{chp:loesungsansaetze} und \ref{chp:systementwurf} jeweils eine Lösung vorgeschlagen und implementiert.

%#######################################################################################