\chapter{Einleitung}\label{chp:einleitung}
\paragraph{}
%Gefordert ist......
Ziel dieser Masterarbeit ist die Erweiterung und Verbesserung des Berner \& Mattner Werkzeuges TESTONA. Dieses Programm bietet Testern ein Werkzeug für eine strukturierte und systematische Ermittlung von Testszenarien\footnote{Kombination mehrerer Testfälle mit dem Ziel, komplexeren Sachverhalten zu überprüfen.}\cite{TESTONA}. Im Kapitel \ref{sec:Testona} wird weiteres zu diesem Programm und die Funktionsweise erläutert.\\

Die Erweiterung des Programmes besteht aus verschiedenen Themen. Ein Thema davon behandelt die Testfallgenerierung und die jeweilige Testabdeckung\footnote{(engl. Test Coverage bzw. Code Coverage) das Verhältnis an tatsächlich getroffenen Aussagen eines Tests gegenüber den theoretisch möglich treffbaren Aussagen. Tests werden anhand der Spezifikation einer zu testenden Software-Einheit definiert.\cite{TestAbdeckung}}. Hier soll garantiert werden, dass bei einer automatischen Testfallgenerierung, eine höchstmögliche Testabdeckung erzielt wird.\\

Die Testfallgenerierung wird in dieser Arbeit beeinflusst, indem die Produktvarianten stärker betrachten werden. Verschiedene Varianten beinhalten verschiedene Parameter und Produktkomponenten (Eigenschaften). Die Parameterwerte definieren auch verschiedene Produktvarianten. Durch das Add-On MERAN für die Anforderungsmanagementsoftware \glqq IBM Rational DOORS\grqq~ können Anforderungen und Varianten direkt in  TESTONA importiert werden. Dabei sollen automatisch die Parameterwerte zur jeweiligen Produktvariante zugeordnet werden. Aus diesem Grund kann es zu Konflikten bei der Testfallgenerierung kommen, bzw. inkohärente Testfälle können auftreten.\\

Um solche Probleme zu vermeiden oder zu umgehen, gibt TESTONA den Testern die Möglichkeit Abhängigkeitsregeln anzulegen. Hier können Anfangsbedingungen sowie Sonderbedingungen definiert werden. Dabei muss wiederum beachtet werden, dass die Produktvarianten nicht verletzt werden. Weiteres zu den Themen und Begriffen wird im Kapitel \ref{chp:fachlichesumfeld} verdeutlicht.\\

Im Kapitel \ref{chp:aufgabenstellung} wird die Aufgabe dieser Masterarbeit genauer erläutert und in den Kapiteln \ref{chp:loesungsansatz} und \ref{chp:systementwurf} jeweils eine Lösung vorgeschlagen und implementiert.

%#######################################################################################